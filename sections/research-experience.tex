\cvsection{Research Experience}
\cveventSameLine{}
    {Strength-Controlled Preference Data Synthesis}
    {Apr. 2024 - Jan. 2025}{\linked[0]{Georgia Tech}{https://www.cc.gatech.edu}}
\begin{itemize}
    % \item Collaborating with PhD student \linked[0]{Rongzhi Zhang}{https://rongzhizhang.org}

    \item Innovating multi-turn preference dataset synthesis with fine-grained preference strengths

    \item Processed 100K samples and filtered low-quality samples by heuristics

    \item Coded \& Optimized distributed training \& inferencing pipeline for 3 distinct reward model architectures, reducing eval time by 2X

    \item Designed uncertainty metrics for training data selection leveraging reward model ensemble logits

    % \item Reward model accuracy trained from negative-prompting-generated data using LLAMA-2-7B matches that of GPT-4 ranking while 30X cheaper
    \item Reward model accuracy trained from negative-prompted data using LLAMA-2-7B matches that of GPT-4 ranking while 30X cheaper
\end{itemize}
\dividerSmall
% \pagebreak


\cveventSameLine{}
    {LLM as IE Data Generator}
    {Sep. 2023 - Mar. 2024}{\linked[0]{Chao Zhang}{http://chaozhang.org}'s group @ \linked[0]{Georgia Tech}{https://www.cc.gatech.edu}}
\begin{itemize}
    % \item Mentored by Ph.D. students at

    \item Innovated 1st structured named-entity recognition training dataset generation with LLMs by attributed prompting for diversity

    \item Developed \& Optimized multi-stage generation pipeline including parallel API calls, data filtering \& cleaning, pretty logging \& summary stats

    \item Manually Inspected generated samples; Case-studied \& Analyzed multiple paradigms for LLM annotation feedback and self-correction

    \item Increased DeBEETa NER F1 score by >5\% on average using \$<1 API cost \& <10 labeled samples,
    matching ChatGPT-3.5 teacher performance while 20X faster; Written 40-page paper with failure analysis; Released package for scalable data generation
\end{itemize}
\dividerSmall


\cveventSameLine{}
    {Parameter-Efficient Personalization}
    {May. 2023 - Dec. 2023}{\linked[0]{CLARITY lab}{https://www.jasonmars.org/2014/03/20/clarity-lab-at-university-of-michigan/} @ \linked[0]{UMich}{https://umich.edu}}{}
\begin{itemize}
    % \item In collaboration with \linked[0]{Christopher Clarke}{https://scholar.google.com/citations?user=IaFEAbsAAAAJ&hl=en}
        % at \linked[0]{CLARITY lab}{https://www.jasonmars.org/2014/03/20/clarity-lab-at-university-of-michigan/}

    \item Surveyed literature to select subjective datasets (e.g., irony)

    \item Designed \& Executed \linked[0]{PEFT}{https://huggingface.co/docs/peft/index}, \linked[0]{Adapter}{https://docs.adapterhub.ml}
    and \linked[0]{Personalized Head}{https://stefanheng.github.io/projects\#PiDset} training \& evaluation user-wise pipelines for Flan-T5 generative text classification

    \item Benchmarked 7 prompting and PEFT methods across 11 subjective tasks, each with up to 5K users and 120K total samples
\end{itemize}
\dividerSmall




\cveventSameLine{}
    {Symbolic Music Generation}
    {Oct. 2021 - Feb. 2023}{\linked[0]{LIT}{https://lit.eecs.umich.edu} @ \linked[0]{UMich}{https://umich.edu}}
\begin{itemize}
    % \item Mentored by \linked[0]{Artem Abzaliev}{http://artem.site44.com}
        % at

    \item Designed \& Implemented a compact music token representation for long song sequences that first integrated
    music theory annotations

    \item Coded \& Optimized tokenization pipeline to process 10K+ raw MIDI files including batching and concurrency optimization, channel reduction and efficient edge-case (>50) handling

    \item Tailored Transformer-XL \& Reformer architectures for long music sequence training; Designed music-specific evaluation metrics; Inspected >100 generated music pieces
\end{itemize}
\dividerSmall




\cveventSameLine{}
    {Personalized Text Classification Dataset}
    {Jul. 2022 - Oct. 2022}{\linked[0]{CLARITY lab}{https://www.jasonmars.org/2014/03/20/clarity-lab-at-university-of-michigan/} @ \linked[0]{UMich}{https://umich.edu}}
\begin{itemize}
    \item Designed a tree-structured text classification dataset schema for nested and temporally-changing label sets

    \item Processed 15K production user data from \linked[0]{Myca}{https://www.myca.ai} productivity tool spanning 2 years

    % \item Implemented and conducted experiments for personalization paper
\end{itemize}
\dividerSmall



\cveventSameLine{}
    {Zero-Shot Text Classification}
    {Feb. 2022 - Jun. 2022}{\linked[0]{CLARITY lab}{https://www.jasonmars.org/2014/03/20/clarity-lab-at-university-of-michigan/} @ \linked[0]{UMich}{https://umich.edu}}
\begin{itemize}
    % \item Mentored by \linked[0]{Christopher Clarke}{https://scholar.google.com/citations?user=IaFEAbsAAAAJ&hl=en}
        % at

    \item Benchmarked 3 zero-shot classification paradigms across 18 datasets

    \item Re-Implemented a closed-sourced, prior GPT-2-based 0-shot approach

    \item Developed \& Optimized training \& eval pipelines, reducing GPT-2 inference time by 2X; Launched experiments

    \item Improved classifier accuracies by 1\% on average with simple domain-conditioned training; Designed illustrations \& wrote paper sections
\end{itemize}
\dividerSmall


\iftoggle{compact}{%
    %
}{%
    \cveventSameLine{}%
        {ECG Classification}%
        {Sep. 2021 - Apr. 2022}{\linked[0]{Michigan Medicine}{https://www.uofmhealth.org} @ \linked[0]{UMich}{https://umich.edu}}%
    \begin{itemize}%
        % \item Mentored by \linked[0]{Dr. Mohammed Saeed}{https://mcircc.umich.edu/members/mohammed-saeed-md-phd} @
        %
        % \item Proposed Machine-Learning-based
        % Premature Ventricular Contraction (PVC) exit site prediction to
        % aid diagnosis \& ablation surgery efficiency
        %
        \item Devised self-supervised pretraining objectives for 12-channel ECG timeseries based on symbolic and real-valued representations%
        %
        \item Reviewed literature to Compile a dataset collection of 50K+ 12-lead ECG records from 8 datasets%
        %
        \item Pioneered applying the Vision Transformer architecture for ECG disease classification with heart-signal data augmentations; Visualized attention layers for explainability%
    \end{itemize}%
    \dividerSmall
    %
    %
    %
    \cveventSameLine{}%
        {ECG Visualization}%
        {Sep. 2020 - Apr. 2021}{\linked[0]{Michigan Medicine}{https://www.uofmhealth.org} @ \linked[0]{UMich}{https://umich.edu}}%
    \begin{itemize}%
        % \item Mentored by \linked{Dr. Mohammed Saeed}{https://mcircc.umich.edu/members/mohammed-saeed-md-phd}
            % at \linked[0]{Michigan Medicine}{https://www.uofmhealth.org}
        %
        \item Developed \linked[0]{Dash}{https://plotly.com/dash/}-based ECG signal web app for
        with features including thumbnail, channel toggle, box measurement \& annotation%
        % for a public cardiovascular disease management and analytics infrastructure
        %
        \item Designed UI wireframes tailored for physicians' retrospective study and annotation needs; Gathered feedbacks from cardiologists%
        %
        \item Algorithmically Optimized rendering efficiency of GBs of signal records%
    \end{itemize}%
    \dividerSmall
    %
    %
    %
    \cveventSameLine{}%
        {AR Art Museum Experience}%
        {Apr. 2019 - May. 2020}{Prof. Kyungjin Yoo @ \linked[0]{UMD}{https://www.fire.umd.edu}}%
    \begin{itemize}%
        % \item Mentored by
        %
        \item Developed an \linked[0]{ARCore}{https://developers.google.com/ar}-based Android app
        that renders art museum paintings in 6 base colors to educate color theory \& raise engagement%
        %
        \item Compared prior approaches for theme-colored visualization; Implemented primary \& secondary color extraction \& color map file parsing%
        %
        \item Developed \& Tested painting segmentation heuristics based on color \& semantics%
    \end{itemize}%
    \dividerSmall
}%
%
%